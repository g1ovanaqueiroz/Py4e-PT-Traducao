% The contents of this file is 
% Copyright (c) 2009- Charles R. Severance, All Righs Reserved

\chapter{Prefácio}

\section*{Python para Informática: Uma abordagem Open Book} %Andhré

É bem comum para os acadêmicos ouvirem frases como ``publique ou pereça`` quando se trata de produzir novas ideias, onde se perpetua uma cultura de começar tudo do zero para que se obtenha uma criação genuína. Este livro é um experimento em não começar da etapa inicial, onde usaremos como base o livro \emph{Pense em Python: Pense como um cientista da computação} escrito por Allen B.Downey, Jeff Elkner, e outros.

Por volta de dezembro de 2009, eu estava me preparando para lecionar
{\bf SI502 - Networked Programming} na University of Michigan pelo quinto semestre seguido, e decidi que estava na hora de escrever um livro de Python focado na exploração de dados ao invés de se limitar ao estudo de algorítimos e abstrações. Meu objetivo naquela turma era para ensinar habilidades vitais utilizando o estudo de dados, para que meus alunos pudessem levar para o resto da vida estes conhecimentos em Python. Poucos destes alunos tinham planos de se tornar cientistas da computação. Como alternativa, eles tinham planos de se tornar economistas, advogados, bibliotecários, biólogos, etc., mas que
mas que queriam usar habilmente a tecnologia e programação nas suas áreas.

Dentro deste contexto, parecia que não havia um livro de Python orientado à análise de dados que se adequasse perfeitamente ao meu curso, e então decidi escrever tal livro. Felizmente, em um encontro na faculdade três semanas antes de começar as férias e consequentemente o início deste projeto, o Professor Dr. Atul Prakash me mostrou o livro \emph{Pense em Python} que ele havia usado para lecionar a disciplina naquele semestre. É um livro bem escrito voltado para ciência da computação e focado em explicações breves, diretas e de fácil compreensão.

A estrutura geral do livro foi modificada para que o leitor possa começar a trabalhar com análise de dados o mais rápido possível, além de ter uma série de exemplos e exercícios desde o começo.

Os capítulos 2--10 são parecidos com os do \emph{Pense em Python}, mas com grandes mudanças. Exercícios com orientação aos números foram substituídos com outros exercícios orientados à análise de dados. Os tópicos são apresentados em uma sequência necessária para evoluir a construção de respostas cada vez mais sofisticadas. Alguns tópicos como {\tt try} e {\tt except} foram colocados mais a frente no capítulo de condicionalidade.  Funções são levemente abordadas no início, até o momento em que seja necessário trabalhar com programas de maior nível de complexidade, ao invés de ser uma abstração inicial. Quase todas as funções que necessitam de definição pelo usuário foram removidas dos códigos de exemplos e exercícios que não sejam do capítulo 4. A palavra ``recursividade''\footnote{Com exceção, é claro, desta linha.} não está presente neste livro de maneira alguma. 

Nos capítulos 1 e 11--16, todo o material apresentado é inédito, com foco em aplicações no mundo real e exemplos simples do uso de Python para a análise de dados, incluindo expressões comuns para pesquisa e análise, automatizando tarefas do seu computador, programação orientada a objetos, recuperando dados por meio da internet, buscando-os em páginas da web, utilizando serviços online, analise de dados XML e JSON, criando e utilizando uma base de dados de Linguagem de Consulta Estruturada (Strutured Query Language - SQL) e visualizando de dados.




O objetivo final destas mudanças é estabelecer uma modificação do foco em ciência da computa-ção para um voltado para informática, incluindo em uma turma inicial de tecnologia apenas tópicos que possam ser úteis mesmo que os alunos não pretendam se tornar programadores profissionais.
%coloquei um - em computação para evitar um erro do latex


Para aqueles que acharem este livro interessante e tiverem a motivação de explorar além dos limites dele, sugiro que deem uma olhada no livro \emph{Pense em Python} do Allen B. Downey. Apesar disso, existem muitas interseções entre os dois livro, e para aqueles que desejam obter habilidades em áreas mais técnicas de programação e construção de algoritmos podem ter acesso a esta informação no livro \emph{Pense em Python}. Dado que os livros possuem uma semelhança no estilo de escrita, a transição entre eles deverá ser fácil e rápida, com o mínimo de esforço.

\index{Creative Commons License}
\index{CC-BY-SA}
\index{BY-SA}
Como proprietário dos direitos autorais do \emph{Pense em Python}, Allen me permitiu modificar a licença do material do livro dele para o material herdado neste livro, da  licença GNU de Documen-tação Livre para a mais recente licença Creative Commons --- licença compartilhável semelhante . Isso acarreta em uma mudança geral na licença de documentação aberta, trocando de uma GFDL para uma CC-BY-SA (ex., Wikipedia). Utilizar a licença CC-BY-SA mantém a forte tradição de direito de cópia (copyleft) ao mesmo tempo em que o processo de novos autores reutilizarem este material como eles acharem melhor se torna mais direto.
%Coloquei um - em Documentação para evitar um erro do latex

Tenho o sentimento de que este livro servirá como um exemplo do porquê materiais com uma compartilhação mais aberta são tão importantes para o futuro da educação, e também gostaria de agradecer ao Allen B. Downey e a gráfica da universidade de Cambridge pela sua decisão voltada para o futuro de tornar este livro disponível sob direitos autorais abertos. Espero que estejam satisfeitos com o resultado dos meus esforços e que você leitor esteja satisfeito com \emph{nossos} esforços coletivos.

Gostaria de agradecer a Allen B. Downey e Lauren Cowles pela ajuda, paciência e orientação em lidar com e resolvendo ativamente problemas relacionados aos direitos autorais deste livro.

\noindent Charles Severance\\
www.dr-chuck.com\\
Ann Arbor, MI, USA\\
 9 de Setembro de 2013

Charles Severance é um professor associado na  \textit{University of Michigan School of Information}.

%\clearemptydoublepage

% TABLE OF CONTENTS
%\begin{latexonly}

%\tableofcontents

%\clearemptydoublepage

%\end{latexonly}

% START THE BOOK
\mainmatter

